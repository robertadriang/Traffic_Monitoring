%% LyX 2.3.4.2 created this file.  For more info, see http://www.lyx.org/.
%% Do not edit unless you really know what you are doing.
\documentclass[12pt,english]{llncs}
\usepackage[T1]{fontenc}
\usepackage[latin9]{inputenc}
\usepackage{graphicx}

\makeatletter

%%%%%%%%%%%%%%%%%%%%%%%%%%%%%% LyX specific LaTeX commands.
%% custom text accent \LyxTextAccent[<rise value (length)>]{<accent>}{<base>}
\newcommand*{\LyxTextAccent}[3][0ex]{%
  \hmode@bgroup\ooalign{\null#3\crcr\hidewidth
  \raise#1\hbox{#2}\hidewidth}\egroup}
%% select a font size smaller than the current font size:
\newcommand{\LyxAccentSize}[1][\sf@size]{%
  \check@mathfonts\fontsize#1\z@\math@fontsfalse\selectfont
}
\ProvideTextCommandDefault{\textcommabelow}[1]{%%
  \LyxTextAccent[-.31ex]{\LyxAccentSize,}{#1}}


\@ifundefined{date}{}{\date{}}
%%%%%%%%%%%%%%%%%%%%%%%%%%%%%% User specified LaTeX commands.
\usepackage{tikz} 
\usepackage{pgfplots}
\usepackage{indentfirst}
\thispagestyle{empty}

\makeatother

\usepackage{babel}
\begin{document}
\title{Traffic monitoring system}
\author{Robert-Adrian G\u{a}in\u{a} (2E4)}
\institute{Faculty of Computer Science ``Alexandru Ioan Cuza'' Ia\textcommabelow{s}i}
\maketitle

\section{Introduction}

Traffic decongestion and fluidization is a modern problem that can
be attenuated with the help of navigation software. This paper serves
as the technical report of the application named Traffic monitoring,
a project recommendation from Continental for the Computer Network
course at Faculty of Computer Science. The scope of this application
is being able to monitorize the traffic upon a certain area by providing
informations to the drivers. In addition to that, the drivers can
report temporary incidents on the road for the others or modify informations
in the system.

This report contains detailed informations of the used technologies,
application architecture, implementation details as well as improvements
that could be added in the future.

\section{Application architecture}

The application will be based upon the \textbf{Client-Server} model
with a \textbf{concurrent server}. In this way, we are avoiding the
starvation problem that appears when using an iterative server with
an increased number of clients. 

The project implementation is based on the Transmission Control Protocol
(\textbf{TCP}). This connection oriented method was chosen in order
to assure a safe transport of data between the clients and the server.
For this type of application we must be sure that we are not losing
any data packets during the transfer nor receiving them in a different
order. In addition to this, it is more reliable to keep a constant
connection between the clients and the server in order to send traffic
updates to the participants. Therefore, the TCP protocol will be used
for implementing the application. 

The concurrency pattern used will be \textbf{threading. }For each
client the server will create a new thread that will execute the commands
required. Threading was chosen over forking because of the common
memory zone present in the technique, allowing the server to send
information faster to the clients. 

The information transmitted between the client and the server will
be done using \textbf{sockets}. Each client will create a parent and
a child process after the connection is established. From this point
the client-server communication is assured by the parent process while
the server-client is managed by the chil. (Write to server in parent
read from child). 

All the traffic details will be stored into a \textbf{SQLite} database.
This database can be modified over time by the clients and each update
of it will be sent to the connected clients.

\section{Implementation details}

\subsection{Traffic mapping}

This app will monitor the traffic of one city split into neighbourhoods.
Each neighbourhoods contains a set of streets with each street containing
a set of informations (F.I. speed limit, sport events, gas stops with
gas prices, weather info, accidents etc...). Any client can report
new incidents on a certain street or neighbourhood that will be registered
into the database. Any incident can be removed from the database if
a number of clients report the incident as a false alert. Every time
a driver enters a new street he will get informations from the server
about the area as well. Additionally, after a fixed period of time
on a street the driver will get the updated information.

\subsection{Estabilishing a connection}

A connection between a client and a server can be established only
by the client. When we want to create a connection the client will
initiate the connection request. When receiving the connection request
the server accepts it and creates a new thread that will receive further
commands from the client or send updates to the client.

\begin{figure}
\includegraphics[scale=0.7]{pasted2}

\caption{Client connecting to the server}

\end{figure}


\subsection{Client-Server communication}

In the application we define two types of communication operation:
\textbf{Interogations} and \textbf{updates}. An interogation is a
request that doesn't alter the database while an update modifies it.

1. The types of interogations are:
\begin{itemize}
\item Data request from a client that changes the neighbourhood/street
\item Data request from a client about a specific zone/information
\item Regular update of informations ( every \textasciitilde 1 minute)
\end{itemize}
\begin{figure}
\includegraphics[scale=0.7]{pasted4}

\caption{An interogation initiated by the client}

\end{figure}

For the first two type of operations the client sends a request to
the server asking for informations. Whenever the localization detects
that the driver changes to a new street/neighbourhood it will send
an automated request to the server asking for informations about the
street/neighbourhood. Additionally, the client can manually send a
request for a certain zone. According to the command received, the
server will search in the database and return the informations required
back to the client. 

All the clients will ask for automatical updates of informations at
every 1 minute. 

2. The types of updates are:
\begin{itemize}
\item A client signals an information that is not in the database already
to the server
\item A set of clients signal the absence of an event that is present in
the database 
\item 
\begin{figure}
\includegraphics[scale=0.7]{pasted5}

\caption{An update triggered by a client}
\end{figure}
\end{itemize}
When a client signals an event that is not registered by the server
the thread responsible with the communication with that client will
modify the database and signal the other clients by using the list
of socket desc riptors about the change. 

Since some of the events are not permanent or fixed (F.I. accidents)
each temporary event can be deleted. When an user comes across a temporary
event he will be notified and allowed to signal the absence of the
event. When a number of consecutive clients report the incident as
being absent it will be removed from the database. For some interogations
the user will also be able to change the data (F.I gas prices) when
near the place. 

\subsection{Use cases examples}

In this section we will present how the application should work in
the final version.

In order to connect each client should already know the adress and
the port where the server allows connection. For every client that
sends a connection request the server will create a thread that should
only respond to that specific client. After the connection is established
the thread will wait for commands.

Connection example:

Two clients send a connection request. $\longrightarrow$ Server creates
two threads

Clients send a request each to the server $\longrightarrow$ The thread
assigned to each client solves the request

One client randomly disconnects $\longrightarrow$ The thread assigned
to that client will close

Commands available to the user: 
\begin{itemize}
\item Login: - connects the user to its account
\item Extra: - allows the user to ask about what type of information he
can receive in traffic. There are 3 levels for this settings. 0- only
important informations, 1- additional events/places on the street,
2-aditional events/places on the neighbourhood where the street is
part of
\item Change: -allows the user to send a change of street/neighbourhood
(This is mostly used automatically by the geolocalization system)
\item Search: -search in the database for events/places based on a pattern.
\item Alert: -allows the user to signal the presence/absence of accidents
or other temporal events. This will insert the event into the database
along with a column that is decremented/incremented based upon the
feedback of other drivers. (If the value becomes 0 the event is dropped
from the database)
\end{itemize}
Alert example:

Client signals an accident that is not registered by the application
on a street $\longrightarrow$ thread responsible for the client adds
the accident in the database as a temporal event (that can be removed
after 3-5 people marked it as absent event) and to a common memory
zone for all the threads to see $\longrightarrow$ all the threads
will pick up the command and send in to their assigned clients $\longrightarrow$
threads will mark that they have notified their clients $\longrightarrow$
when all threads notified their clients the accident will be removed
from the common memory zone.

Elimination of alert example:

Client asks about informations upon a street where an accident was
recently reported $\longrightarrow$ The server will send the related
informations about the street and ask for the confirmation of the
accident$\longrightarrow$ if the accident is marked by the client
as absent the server will decrement the value detailed before otherwise
increment $\longrightarrow$ If the value is 0 the accident is dropped
from the database.

\section{Conclusion}

This application can help mitigating the daily traffic inconveniences
and also signaling flawed street networks. 

This application could benefit from the following improvements:
\begin{itemize}
\item Being able to configure a starting and ending point in order to send
alerts only present on the route planned
\item The updates/informations could be sent only if they are from the neighbourhood
where the driver currently is.
\item A graphic interface to ease the experience of the user.
\item Implementing street network flows on the map allowing us to see how
crowded it is.
\item A localization system.
\item A gps speedometer system.
\end{itemize}
\begin{thebibliography}{1}
\bibitem{key-3}General informations about computer networks\\
https://profs.info.uaic.ro/\textasciitilde computernetworks/cursullaboratorul.php

\bibitem{key-7}Project description {[}Monitorizarea traficului (A)
{[}Propunere Continental{]}{]}\\
https://profs.info.uaic.ro/\textasciitilde computernetworks/ProiecteNet2020.php

\bibitem{key-9}TCP vs UDP\\
https://www.guru99.com/tcp-vs-udp-understanding-the-difference.html

\bibitem{key-11}Thread vs Fork\\
http://www.geekride.com/fork-forking-vs-threading-thread-linux-kernel

\bibitem{key-12}Waze (Application functionality inspiration)\\
https://www.waze.com/
\end{thebibliography}

\end{document}
